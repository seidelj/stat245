\documentclass{tufte-book}

\usepackage{amsmath, amsthm}
\usepackage{graphicx}
\setkeys{Gin}{width=\linewidth,totalheight=\textheight,keepaspectratio}
\graphicspath{{graphics/}}

\title{STAT  245\\Homework 0}
\author{Joe Seidel}
\date{\today}


\usepackage{booktabs}
\usepackage{units}
\usepackage{fancyvrb}
\fvset{fontsize=\normalsize}
\usepackage{multicol}
\usepackage{lipsum}
\usepackage{pdfpages}
\usepackage{tikz}
\usepackage{wasysym}
\usepackage{amssymb}
\usepackage{hyperref}
\usepackage{float}
\restylefloat{table}

\hypersetup{
    colorlinks=true,
    linkcolor=blue,
    filecolor=magenta,
    urlcolor=cyan,
}


\newcommand{\doccmd}[1]{\texttt{\textbackslash#1}}% command name -- adds backslash automatically
\newcommand{\docopt}[1]{\ensuremath{\langle}\textrm{\textit{#1}}\ensuremath{\rangle}}% optional command argument
\newcommand{\docarg}[1]{\textrm{\textit{#1}}}% (required) command argument
\newenvironment{docspec}{\begin{quote}\noindent}{\end{quote}}% command specification environment
\newcommand{\docenv}[1]{\textsf{#1}}% environment name
\newcommand{\docpkg}[1]{\texttt{#1}}% package name
\newcommand{\doccls}[1]{\texttt{#1}}% document class name
\newcommand{\docclsopt}[1]{\texttt{#1}}% document class option name
\DeclareMathOperator{\proj}{proj}
\newcommand{\vct}{\mathbf}
\newcommand{\dprod}[2]{\langle \vct{#1}, \vct{#2} \rangle}
\newcommand{\pdv}[2]{\frac{\partial #1}{\partial #2}}
\DeclareMathOperator{\Var}{Var}
\DeclareMathOperator{\Cov}{Cov}
\newtheoremstyle{mytheoremstyle} % name
	{\topsep}		% Space above
	{\topsep}		% Space below
	{\itshape}		% Body font
	{}			% Indent amount
	{\bfseries}	% Theorem head font
	{\textnormal{:}}	% Punctuation after theorem head
	{.5em}		% Space after theorem head
	{}			%Theorem headspec
\theoremstyle{mytheoremstyle}
\newtheorem*{thm}{Thm.}

\newtheoremstyle{mylemstyle} % name
	{\topsep}		% Space above
	{\topsep}		% Space below
	{\itshape}		% Body font
	{}			% Indent amount
	{\bfseries}	% Theorem head font
	{\textnormal{:}}	% Punctuation after theorem head
	{.5em}		% Space after theorem head
	{}			%Theorem headspec
\theoremstyle{mylemstyle}
\newtheorem*{lem}{Lem.}


\newtheoremstyle{mydefstyle} % name
	{\topsep}		% Space above
	{\topsep}		% Space below
	{\normalfont}	% Body font
	{}			% Indent amount
	{\bfseries}	% Theorem head font
	{\textnormal{:}}	% Punctuation after theorem head
	{.5em}		% Space after theorem head
	{}			%Theorem headspec
\theoremstyle{mydefstyle}
\newtheorem*{mydef}{Def.}
\newtheorem*{ex}{E.g.}


\begin{document}

\maketitle
\pagenumbering{gobble}
\newpage
\pagenumbering{arabic}

\subsection{1. Approximate confidence intervals for Poisson Distribution}
Let $X_1,...,X_n$ be independent random variables distributed according to a Poison$(\lambda)$ distribution.  Then the MLE of $\lambda$ is $\hat{\lambda}=\overline{X}= \frac{1}{n}\sum_{i=1}^nX_i$, and the two r.v.

\[ \frac{\hat{\lambda} - \lambda}{\sqrt{\lambda/n}}\text{ and } \frac{\hat{\lambda} - \lambda}{\sqrt{\hat{\lambda}/n}} \]

both have approximately $N(0,1)$ for large $n$.  Using the "pivotal method" derive two approximate confidence intervals for $\lambda$.  What are the interval midpoints?  Are the intervals guaranteed to comprise only nonegative numbers?  Explain.

For $\frac{\hat{\lambda} - \lambda}{\sqrt{\lambda/n}}$ see Prof. Gao's handout using Wilson's approach.

For $\frac{\hat{\lambda} - \lambda}{\sqrt{\hat{\lambda}/n}}$ observe

\[ \frac{\hat{\lambda} - \lambda}{\sqrt{\hat{\lambda}/n}}  \rightsquigarrow N(0,1) \]

is assymptotic pivotal and the CLT implies

\begin{equation}
\Pr\big( z_{\frac{\alpha}{2}} \leq \frac{\sqrt{n}(\hat{\lambda}-\lambda)}{\sqrt{\hat{\lambda}}} \leq z_{1-\frac{\alpha}{2}} \big) \approx 1-\alpha.
\end{equation}

The inequality in equation $(1)$ can be manipulated

\begin{align*}
z_{\frac{\alpha}{2}} &\leq \frac{\sqrt{n}(\hat{\lambda}-\lambda)}{\sqrt{\hat{\lambda}}} \leq z_{1-\frac{\alpha}{2}}\\
z_{\frac{\alpha}{2}}\frac{\sqrt{\hat{\lambda}}}{\sqrt{n}} &\leq (\hat{\lambda}-\lambda) \leq z_{1-\frac{\alpha}{2}}\frac{\sqrt{\hat{\lambda}}}{\sqrt{n}}\\
\hat{\lambda} - z_{1-\frac{\alpha}{2}}\frac{\sqrt{\hat{\lambda}}}{\sqrt{n}} &\leq \lambda \leq \hat{\lambda} + z_{1-\frac{\alpha}{2}}\frac{\sqrt{\hat{\lambda}}}{\sqrt{n}}\\
\end{align*}.

Hence

\[ \Pr(\hat{\lambda} - z_{1-\frac{\alpha}{2}}\frac{\sqrt{\hat{\lambda}}}{\sqrt{n}} \leq \lambda \leq \hat{\lambda} + z_{1-\frac{\alpha}{2}}\frac{\sqrt{\hat{\lambda}}}{\sqrt{n}}) \approx 1-\alpha \]

and the confidence interval is
\[ [ \hat{\lambda} \pm z_{1-\frac{\alpha}{2}}\frac{\sqrt{\hat{\lambda}}}{\sqrt{n}} ] \]

with midpoint $\hat{\lambda}=\overline{X}$.  Furthermore, the interval does not guarantee comprising non-negative values.  Consider $\hat{\lambda}=1$ and small $n$.

Running \marginnote{R code available q1.R} $100000$, with $n=30$, $\alpha=.05$ and $\lambda=1$, Wilson's confidence iterval does slightly better, $.9751$ vs $.9291$.


\subsection{2. Sample size determination}
Let $X$ follow a Binomial$(n,p)$ distribution and let $\hat{p}=\frac{\overline{X}}{n}$ be the maximum likelihood estimator of the success probability, $p$. Recall that the "Wald" $(1-\alpha)100\%$ confidence interval for $p$ is of the form

\[ [\hat{L}, \hat{U}] = \Big[ \hat{p} \pm z_{1-\frac{\alpha}{2}} \sqrt{\frac{\hat{p}(1-\hat{p})}{n}} \Big]. \]

For $\alpha=.05$ find the smalled integer $n_0 \in \mathbb{N}$ such that for all $n \geq n_0$ the confidence interval has length $\hat{U} - \hat{L} \leq 0.06$ regardless of the value $p \in [0,1]$.

Through algebra observe

\begin{align*}
\hat{U} - \hat{L} &= \big(\hat{p} + 1.96 \sqrt{\frac{\hat{p}(1-\hat{p})}{n}}\big) - \big(\hat{p} + 1.96 \sqrt{\frac{\hat{p}(1-\hat{p})}{n}}\big)\\
&= 3.92 \sqrt{\frac{\hat{p}(1-\hat{p})}{n}}
\end{align*}

then

\begin{align*}
3.92 \sqrt{\frac{\hat{p}(1-\hat{p})}{n}} &\leq 0.06\\
\frac{\hat{p}(1-\hat{p})}{n} &\leq .01515^2\\
n &\geq \frac{\hat{p}(1-\hat{p})}{0.01515^2}
\end{align*}

Since $\hat{p}(1-\hat{p})$ is largest when $\hat{p}=.5$ we should use that value in the above inequality and conlude $n\geq 1098$.


\subsection{3. Approximate confidence intervals for Binomial distribution}
Let $X$ have Binomial$(n,p)$ distribution, and let $\hat{p}=\frac{X}{n}$ be the maximum likelihood estimator of the success probability $p$.

For the "Wald method", "Wilson method", and the arcsin transformation simulate in R. What proportion of the confidence intervals would we expect to contain $p=.1$ if the approximations are good.  From simulations, which proportion of confidence intervals actually contain $p=.1$.

Given $\alpha=0.05$ expect that $\frac{95}{100}$ of the intervals contain $p=0.1$ if approximations are good.

Running $n=100$ simulations, calculate the intervals and the proportions that contain $p=.1$.
\begin{enumerate}

\item The Wald interval

\[ \Big[ \hat{p} \pm z_{1-\frac{\alpha}{2}} \sqrt{\frac{\hat{p}(1-\hat{p})}{n}} \Big] \]

Approximately $83\%$ of the confidence intervals contained $p=.1$.

\item The Wilson interval

\[ \Big[ \frac{ \hat{p} + \frac{z^2}{2n} \pm \sqrt{\frac{\hat{p}(1-\hat{p})}{n}z^2 + \frac{z^4}{4n^2}} }{1 + \frac{z^2}{n}} \Big] \text{ with } z=z_{1-\frac{\alpha}{2}} \]

Approximately $97\%$ of confidence intervals simulated contained $p=.1$.

\item The arcsin transformation

\[ \Big[ \sin^2(\arcsin(\sqrt{p}\pm \frac{ z_{1-\frac{\alpha}{2}}}{2\sqrt{n}})) \Big] \]

Approximately $93\%$ of confidene intervals contain $p=.1$.

\end{enumerate}

By repeting with $n=150$ the propertions get closer to $95\%$.

\subsection{4. Distribution of a ratio}
Show that if $X_1$ and $X_2$ are independent exponential random variables with parameter $\lambda=1$, then $\frac{X_1}{X_2}$ follows an F-distribution.  Also identify the degress of freedom.

Observe
\begin{align*}
f_X &= f_{X_1} = e^{-x} \\
f_Y &= f_{X_2} = e^{-y} \\
\end{align*}

Let $U=X$ and $V=\frac{X}{Y}$ then $X=U$ and $Y=\frac{U}{V}=g(x)$.

\[ f_{U,V} = f_{X,Y}(u, g(x))|g'(x)| = e^{-u(1+\frac{1}{v})}(\frac{u}{v^2}). \]

Then, find the marginal distribution of $f_V$ by integrating out $U$.
\begin{align*}
f_V(v) &= \int_{0}^{\infty}e^{-u(1+\frac{1}{v})}(\frac{u}{v^2})du \\
&= \frac{1}{v^2} \int_{0}^{\infty}e^{-u(1+\frac{1}{v})}u\frac{1+\frac{1}{v})^2}{\Gamma(2)}du\frac{\Gamma(2)}{(1+\frac{1}{v})^2}\\
&= (\frac{1}{v^2}) (1+\frac{1}{v})^{-2}\\
&= (1+v)^{-2} \sim F_{2,2}
\end{align*}

\subsection{5. Do questions 16, 17, and 18 on p.241 in Rice}

\begin{enumerate}

\item True or False?

The center of a $95\%$ confidence interval for the population mean is a random variable. TRUE

A $95\%$ confidence interval for $\mu$ contains the sample mean with probability $95\%$.  FALSE: the interval is build around the sample mean so it it contains with probability 1.

A $95\%$ confidence interval contains $95\%$ of the population. FALSE: A CI means that some percentange of samples constructed using indentical methods will contain the true parameter.

Out of one hundred $95\%$ confidence intervals for $\mu$, 95 will contain $\mu$. FALSE: It is actually a Binom$(100, .95)$ random variable.

\item A $90\$$ confidence interval for the average number of children per house based on a simple random sample is fund to be $(.7, 2.1)$.  Can we conlcude that $90\%$ of households have between $.7$ and $2.1$ children?

No. The correct interpretation of the interval would be: were the sample procedure repeated on numerous samples the fraction of calculated intervals that contain the true mean would tend toward $95\%$.


\item From independent surveys of two populations, $90\%$ confidence intervals for the population means are constructed.  What is the probability that neither interval contains the respective population mean?  That both do?

For both $\binom{2}{2}.9^2$.  For niether $\binom{2}{0}.9^0(1-.9)^{.2}$.

\end{enumerate}

\subsection{6. Pivotal quantities and Normal distribution}
Let $X_1, ...,X_n$ be iid as $N(\mu, \mu^2)$,where $\mu \in \mathbb{R}$ is an unknown parameter.

\begin{enumerate}

\item[(a)] Find pivotal(s) for $\mu$.

Since the MLE, $\hat{\mu}=\overline{X}$ then

\[ \frac{\sqrt{n}(\overline{X} - \mu)}{\mu} = \sqrt{n}\Big(\frac{\overline{X}}{\mu} - 1\Big) \sim N(0,1) \]
is a pivotal for $\mu$.

Also, since the sample variance is

\[ s^2 = \frac{1}{n-1} \sum_{i=1}^n(X_i - \overline{X})^2 \]

and a random variable we have the following result

\[ \frac{\sum_{i=1}^n(X_i -\overline{X})^2}{\mu^2} = \frac{(n-1)s^2}{\mu^2} \sim \chi_{n-1}^2. \]

Another pivotal quantity for $\mu$.  There may be more, I do not know at this point.

\item[(b)] Let $\hat{\mu}$ be the MLE for $\mu$.  Find a function $g$ such that \[ \sqrt{n}|g(\hat{\mu}-g(\mu)| \Rightarrow N(0,1). \]

First consider the likelihood function

\[ f(\mu \ | \ X_1,...,X_n) = \Big(\frac{1}{\mu\sqrt{2\pi}}\Big)^n e^{-\sum_{i=1}^n\Big(\frac{(X_i-\mu)^2}{2\mu^2}\Big)} \]

and
\[ l(\mu \ | \ X_1,...,X_n) = -n \log(\mu) - \frac{n}{2}\log(2\pi) - \frac{\sum_{i=1}^n(X_i-\mu)^2}{2\mu^2}. \]

Where
\[ \pdv{l}{\mu} = -\frac{n}{\mu} + \sum_{i=1}^n\frac{x_i^2}{\mu^3} - \sum_{i=1}^n\frac{x_i}{\mu^2} \]

which when set to $0$ gives

\[ n\mu^2 + \mu\sum_{i=1}^n x_i -\sum_{i=1}^n x_i^2 = 0 \]

whose positive root will be the MLE of $\mu$.

To find $g$, use Talyor expansion in conjunction with asymptotic normaliy.  If there exists $g$ such that
\[ \sqrt{n}[g(\hat{\mu})-g(\mu)] \rightarrow N(0,1) \]
then by Taylor expansion
\begin{equation} \sqrt{n}g'(\mu)(\hat{\mu}-\mu) \approx \sqrt{n}(g(\hat{\mu})-g(\mu)) \rightarrow N(0,1).
\end{equation}

By asymptotic normalily we have
\begin{equation}
\sqrt{nI(\mu)}(\hat{\mu} - \mu) \rightarrow N(0,1)
\end{equation}

where
\begin{align*}
I(\mu) &= -E\pdv{^2l}{\mu^2}\\
&= -\frac{1}{\mu^2} + \frac{3}{\mu^4}E(X^2) - \frac{2}{\mu^3}E(X)\\
&= \frac{3}{\mu^2}
\end{align*}
Fisher information.

Compairing left hands sides of $(2)$ and $(3)$
\[ g'(\mu) = \frac{\sqrt{3}}{\mu} \]
which implies $g(u) = \sqrt{3}\log(\mu)$.

\item[(c)] Comment on the confidence intervals for $\mu^2$ constructed based on (a) and (b).  Which one has smaller length.

The interval from (b) is smaller. \marginnote{I got lazy and didn't derive this.  See the solutions from thr TA}

\end{enumerate}
\end{document}
\grid
