\documentclass{tufte-book}

\usepackage{amsmath, amsthm}
\usepackage{graphicx}
\setkeys{Gin}{width=\linewidth,totalheight=\textheight,keepaspectratio}
\graphicspath{{graphics/}}

\title{STAT  245\\Homework 2}
\author{Joe Seidel}
\date{\today}


\usepackage{booktabs}
\usepackage{units}
\usepackage{fancyvrb}
\fvset{fontsize=\normalsize}
\usepackage{multicol}
\usepackage{lipsum}
\usepackage{pdfpages}
\usepackage{tikz}
\usepackage{wasysym}
\usepackage{amssymb}
\usepackage{hyperref}
\usepackage{float}
\restylefloat{table}

\hypersetup{
    colorlinks=true,
    linkcolor=blue,
    filecolor=magenta,
    urlcolor=cyan,
}


\newcommand{\doccmd}[1]{\texttt{\textbackslash#1}}% command name -- adds backslash automatically
\newcommand{\docopt}[1]{\ensuremath{\langle}\textrm{\textit{#1}}\ensuremath{\rangle}}% optional command argument
\newcommand{\docarg}[1]{\textrm{\textit{#1}}}% (required) command argument
\newenvironment{docspec}{\begin{quote}\noindent}{\end{quote}}% command specification environment
\newcommand{\docenv}[1]{\textsf{#1}}% environment name
\newcommand{\docpkg}[1]{\texttt{#1}}% package name
\newcommand{\doccls}[1]{\texttt{#1}}% document class name
\newcommand{\docclsopt}[1]{\texttt{#1}}% document class option name
\DeclareMathOperator{\proj}{proj}
\newcommand{\vct}{\mathbf}
\newcommand{\dprod}[2]{\langle \vct{#1}, \vct{#2} \rangle}
\newcommand{\pdv}[2]{\frac{\partial #1}{\partial #2}}
\DeclareMathOperator{\Var}{Var}
\DeclareMathOperator{\Cov}{Cov}
\newtheoremstyle{mytheoremstyle} % name
	{\topsep}		% Space above
	{\topsep}		% Space below
	{\itshape}		% Body font
	{}			% Indent amount
	{\bfseries}	% Theorem head font
	{\textnormal{:}}	% Punctuation after theorem head
	{.5em}		% Space after theorem head
	{}			%Theorem headspec
\theoremstyle{mytheoremstyle}
\newtheorem*{thm}{Thm.}

\newtheoremstyle{mylemstyle} % name
	{\topsep}		% Space above
	{\topsep}		% Space below
	{\itshape}		% Body font
	{}			% Indent amount
	{\bfseries}	% Theorem head font
	{\textnormal{:}}	% Punctuation after theorem head
	{.5em}		% Space after theorem head
	{}			%Theorem headspec
\theoremstyle{mylemstyle}
\newtheorem*{lem}{Lem.}


\newtheoremstyle{mydefstyle} % name
	{\topsep}		% Space above
	{\topsep}		% Space below
	{\normalfont}	% Body font
	{}			% Indent amount
	{\bfseries}	% Theorem head font
	{\textnormal{:}}	% Punctuation after theorem head
	{.5em}		% Space after theorem head
	{}			%Theorem headspec
\theoremstyle{mydefstyle}
\newtheorem*{mydef}{Def.}
\newtheorem*{ex}{E.g.}


\begin{document}

\maketitle
\pagenumbering{gobble}
\newpage
\pagenumbering{arabic}

\subsection{Confidence intervals}
Let $X_1,...,X_n$ be independent $N(\mu, \sigma^2)$ random variables.

\begin{enumerate}
\item[(a)] Determine random variables $L_{\sigma^2}(\alpha)$ and $U_{\sigma^2}(\alpha)$ such that the interval $[L_{\sigma^2}(\alpha),U_{\sigma^2}(\alpha)$ is a $(1-\alpha)$ confidence interval for $\sigma^2$.  In doing this ensure that
\[ \Pr(L_{\sigma^2}>\sigma^2) = \Pr(U_{\sigma^2} < \sigma^2) \]

Consider two estimators for $\sigma^2$
\[ s^2 = \frac{1}{n-1}\sum_{i=1}^n(X_i-\overline{X})^2 \]
and
\[ \hat{\sigma}^2 = \frac{1}{n}\sum_{i=1}^n(X_i - \overline{X})^2. \]

Since
\[ \frac{(n-1)s^2}{\sigma^2} \sim \chi^2_{n-1}. \]
then
\[ \frac{n\hat{\sigma}^2}{\sigma^2} \sim \chi_{n-1}^2 \]
and
\[ \Pr(\chi_{n-1}^2(1-\alpha/2) \leq \frac{n\hat{\sigma}^2}{\sigma^2} \leq \chi_{n-1}^2(\alpha/2)) = 1-\alpha \]
where $\chi_{m}^2(\alpha)$ denotes the point beyond which the chi-squared distribution with $m$ degrees of freedom has probabilty $\alpha$.  With some manipulation
\[ \Pr\big(\frac{n\hat{\sigma}^2}{\chi_{n-1}^2(\alpha/2)}\leq \sigma^2 \leq \frac{n\hat{\sigma}^2}{\chi_{n-1}^2(1-\alpha/2)}\big) = 1-\alpha \]

hence the interval for $\sigma^2$ is
\[ \big[\frac{n\hat{\sigma}^2}{\chi_{n-1}^2(\alpha/2)},\frac{n\hat{\sigma}^2}{\chi_{n-1}^2(1-\alpha/2)}\Big]. \]

\item[(b)] Find $(1-\alpha)100\%$ confidence interval for $\mu$ using t-statistic.

We can assume that $\hat{\mu} \sim N(\mu, \frac{s^2}{n})$, approximately from CLT.
Since
\[ \frac{\sqrt{n}(\overline{X}-\mu)}{s} \sim t_{n-1} \]
where $s^2$ is as defined in part $(a)$ then we can build an interval for $\overline{X}$ without knowing $\sigma^2$.  Let $t_m(\alpha)$ denote that point beyond which the t-distribution with $m$ degrees of freedom has probabilty $\alpha$. By symmetry of t
\[ \Pr(-t_{n-1}(\alpha/2) \leq \frac{\sqrt{n}(\overline{X}-\mu)}{s} \leq t_{n-1}(\alpha/2) ) = 1-\alpha \]

which can be manipulated to
\[ \Pr( \overline{X}-\frac{s}{\sqrt{n}}t_{n-1}(\alpha/2)\leq \mu \leq \overline{X}+\frac{s}{\sqrt{n}}t_{n-1}(\alpha/2)) = 1-\alpha \]
forming the interval
\[ \big[ \overline{X} \pm \frac{s}{\sqrt{n}}t_{n-1}(\alpha/2) \big]. \]

\item[(c)] What is the probability that both $\mu \in [L_\mu(\alpha), U_\mu(\alpha)]$ and $\sigma^2 \in [L_{\sigma^2}(\alpha), U_{\sigma^2}(\alpha)]$.

Since $s^2$ appears in both pairs of endpoints, there is no reason to think that the intervals are independent.  The event that $\mu \in [L_{\mu}(\alpha), U_\mu(\alpha)]$ and $\sigma^2 \in [L_{\sigma^2}(\alpha), U_{\sigma^2}(\alpha)]$ can be represented as the region in $\overline{X} - s$ plane bound by the following lines
\begin{align*}
L_1:& \ s = \sqrt{\sigma^2 \chi_{n-1}^2(\alpha/2)/(n-1)}\\
L_2:& \ s = \sqrt{\sigma^2 \chi_{n-1}^2(1-\alpha/2)/(n-1)}\\
L_3:& \ \overline{X} = \mu + \frac{s}{\sqrt{n}}t_{n-1}(\alpha/2)\\
L_4:& \ \overline{X} = \mu + \frac{s}{\sqrt{n}}t_{n-1}(1-\alpha/2).
\end{align*}

Then
\begin{align*}
\Pr&(\mu \in [L_{\mu}(\alpha), U_\mu(\alpha)], \sigma^2 \in [L_{\sigma^2}(\alpha), U_{\sigma^2}(\alpha)])\\
&= 1 -\Pr(\mu \not\in [L_{\mu}(\alpha), U_\mu(\alpha)] \text{ or } \sigma^2 \not\in [L_{\sigma^2}(\alpha), U_{\sigma^2}(\alpha)])\\
&\geq 1 - \Pr(\mu \in [L_{\mu}(\alpha), U_\mu(\alpha)])-  \Pr(\sigma^2 \in [L_{\sigma^2}(\alpha), U_{\sigma^2}(\alpha)])\\
&\geq 1 - 2\alpha
\end{align*}

\item[(d)] Certainly there exist many confidence intervals for $\sigma^2$, and the answer in (a) is one of them.  Find the confidence interval which has minimum length.

Since the Chi-squared distribution is not symmetric for small values of $m$, you need find points on the distribution $a,b$ such that $f(a)=f(b)$ and $F(b)-F(a)=0$.  There is some R code in the directory that will solve if degress of freedom are provided.

To obtain a different confidence interval choose $0<a<b$ such that
\[ \Pr(a < \frac{(n-1)s^2}{\sigma^2} < b) = 1-\alpha \rightarrow (*). \]

Therefore
\[ \Big( \frac{(n-1)s^2}{b}, \frac{(n-1)s^2}{a} \Big)\]
is a $100(1-\alpha)\%$ confidence interval for $\sigma^2$ with length of \[(n-1)s^2\big(\frac{1}{a} - \frac{1}{b}\big).\]

We need to find a choice $a$ and $b$ that minimizes $\frac{1}{a} - \frac{1}{b}$ subject to the constraint $(*)$. Since $(*)$ determines $b$ as a function of $a$ we can use implicit differentiation:
\[ \frac{d}{da} \int_{a}^{b} f_{n-1}(x)dx = \frac{d}{da}(1-\alpha)\]
where $f_{n-1}$ is the density $\chi_{n-1}^2$.   Then
\[ -f_{n-1}(a) + f_{n-1}(b)\frac{db}{da} = 0 \Rightarrow \frac{db}{da}= \frac{f_{n-1}(a)}{f_{n-1}(b)}. \]

Now
\[ \frac{d}{da}\big(\frac{1}{a} - \frac{1}{b}\big) = -\frac{1}{a^2} + \frac{1}{b^2}\frac{f_{n-1}(a)}{f_{n-1}(b)}\]
which equals $0$ when $a^2f_{n-1}(a) = b^2f_{n-1}(b)$.  This result can be verified as a minumum thus an interval of mininum length when $a,b$ such that
\[ a^2f_{n-1}(a) = b^2f_{n-1}(b) \]
\end{enumerate}

\subsection{2. Likelihood ratio test}
Consider i.i.d. observations $X_1,...,X_n \sim p_{\theta}$.  One needs to test
\[ H_0  : \ \theta = \theta_0 \ , \ H_1  : \ \theta=\theta_1. \]

The likelihood ratio test is

\[ \mathbb{I}\Big\{\prod_{i=1}^n \frac{p_{\theta_1}(X_i)}{p_{\theta_0}(X_i)} > C \Big\}, \]

where C is some number such that the Type-I error is $\alpha$.

\begin{enumerate}
\item[(a)] Find the $\alpha$-level likelihood ratio test for $H_0: \ \theta=5$ vs $H_1: \ \theta=10$ with observations $X_1,...,X_n sim N(\theta,1)$.

Reject $H_0$ when $\overline{X} > C$, e.g. large values of $\overline{X}$ should be considered evidence for $H_1$\marginnote{Stigler 6-6 derives where this idea comes from.}.

Let
\begin{align*}
\alpha &= \Pr(\overline{X} > C \mid H_0)\\
&= \Pr(\frac{\sqrt{n}(\overline{X}-\mu_0)}{\sigma_0} > \frac{\sqrt{n}(C-\mu_0)}{\sigma_0} \mid H_0)\\
&= \Pr(Z > \frac{\sqrt{n}(C-\mu_0)}{\sigma_0}
\end{align*}
Where $Z \sim N(0,1)$. To find $C$, observe
\[ z_{1-\alpha} = \frac{\sqrt{n}(C-\mu_0)}{\sigma_0} \Rightarrow C = \mu_0 + z_{1-\alpha}\big(\frac{\sigma_0}{\sqrt{n}}\big). \]

We know $\mu_0=5$ and $\sigma_0=\sigma=1$ so

\[ C = 5 + z_{1-\alpha}\frac{1}{\sqrt{n}}. \]

The test will reject $H_0$ when $\overline{X} > C$ for observed $X_1,...,X_n$ and given $\alpha$.

\item[(b)] What is the power of the test when $n=10,50,100$?
Power is
\[ \pi = \Pr(\overline{X}>C \mid H_1) = 1-\beta\]
where
\begin{align*}
\beta &= \Pr(\overline{X} \leq C \mid H_1)\\
&= \Pr(\overline{X} - \mu_1 \leq C-\mu_1)\\
&= \Pr(Z \leq (5+z_{1-\alpha}\frac{1}{\sqrt{n}})-10))\\
&= \Pr(Z \leq -5 + z_{1-\alpha}(\frac{1}{\sqrt{n}}))
\end{align*}

Where $\beta \rightarrow pnorm(-5)$ as $n \rightarrow \infty$ so Power increases as $n$ increases.

\item[(c)] Find the $\alpha$-level likelihood ratio test for $H_0: \ \theta=0$ vs $H_1: \ \theta = n^{-1/2}$ with observations $X_1,...,X_n \sim N(\theta, 1)$,  What is the power of the test when $n=10,50,100$?

Observe, where $n^{-1/2}=\frac{1}{\sqrt{n}}$ we reject $H_0$ when $\overline{X} > C$ where
\[ C= z_{1-\alpha}(\frac{1}{\sqrt{n}}). \]
The power of this test is $1-\beta$ where
\begin{align*}
\beta &= \Pr(\overline{X} \leq C \mid H_1)\\
&=\Pr(Z \leq \frac{z{1-\alpha}}{\sqrt{n}} - \frac{1}{\sqrt{n}}) \\
&= \Pr(Z \leq \frac{1}{\sqrt{n}}(z_{1-\alpha} - 1 )).
\end{align*}

As $n \rightarrow \infty$ $\beta \rightarrow .5$.

\item[(d)] Discuse your discovery.

Power and $H_1$ both depend on $n$ smaple size.  When $n$ increases, power decreases to $.5$ because it becomes harder to discern $H_0$ from $H_1$.

\subsection{3. Chi-squared}
Consider i.i.d. observations $X_1,...,X_n \sim N(\mu, \sigma^2)$.  The mean $\mu$ is unknown.  Construct an $\alpha$-level test for $H_0: \ \sigma^2=\sigma_0^2$ vs $H_1: \ \sigma^2 \neq \sigma_0^2$.

The test statistic will be

\[ \frac{(n-1)s^2}{\sigma^2_0} \sim \chi_{n-1}^2. \]

Reject $H_0$ if $\chi_{n-1}^2 \in W$ where $W=[0, z_1) \cup (z_2, \infty]$ and $z_1,z_2 \in \mathbb{R}$.  To determine critical values, $z_1,z_2$ use Chi-squared table with $n-1$ degrees of freedom to satisfy
\[ \alpha = \Pr(\frac{(n-1)s^2}{\sigma^2_0} \in W) \]
alternatively
\[ 1-\alpha = \Pr(\frac{(n-1)s^2}{\sigma^2_0} \not\in W) \]
which is similar to confidence interval found in question 1.
\[ 1-\alpha = \Pr\big( \frac{(n-1)s^2}{\chi_{n-1}^2(\alpha/2)} \leq \sigma_0^2 \leq \frac{(n-1)s^2}{\chi_{n-1}^2(1-\alpha/2)}\big). \]

Which can be written as confidence interval
\[  \big[\frac{(n-1)s^2}{\chi_{n-1}^2(\alpha/2)},  \frac{(n-1)s^2}{\chi_{n-1}^2(1-\alpha/2)} \big]. \]

Therefore, our test rejects $H_0$ if $\sigma_0^2$ not in the CI.

Specifically stated, reject the null when
\[ \frac{(n-1)s^2}{\sigma_0^2} < \chi_{n-1}^2(\alpha/2) \text{ or } \frac{(n-1)s^2}{\sigma_0^2} > \chi_{n-1}^2(1-\alpha/2) \]
where $\chi_{n-1}^2(a)$ is the $a$ percentile of of a Chi-squared distribution with $n-1$ degress of freedom.
\end{enumerate}

\subsection{4. Problem 1 on Page 362}
A coin is thrown independently 10 times to test the hypothesis that the probability of heads is $\frac{1}{2}$ versus the alternative that the probability is not $\frac{1}{2}$.  The test rejects if either $0$ or $10$ heads are observed.

\begin{enumerate}

\item[(a)] What is the significance level of the test?

\begin{align*}
\alpha &= \Pr(\text{Reject }H_0 \mid H_0)\\
&= \Pr(X=0 \text{ or } X=10 \mid H_0)\\
&= \Pr(X=0) + \Pr(X=10)\\
&= \binom{10}{0}.5^0(.5)^{10} + \binom{10}{10}.5^{10}(1-.5)^0\\
&= .00195
\end{align*}

\item[(b)] If in fact the probability of heads is $.1$, what is the power of the test?

\begin{align*}
\pi &= 1-\beta = \Pr(\text{Reject }H_0 \mid p=.1)\\
&= \Pr(X=0 \text{ or } X=10 \mid p=.1)\\
&= \binom{10}{0}.1^0(1-.1)^{10} + \binom{10}{10}.1^{10}(1-.1)^0\\
&= .349
\end{align*}


\end{enumerate}


\end{document}
\grid
