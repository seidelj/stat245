\documentclass{tufte-book}

\usepackage{amsmath, amsthm}
\usepackage{graphicx}
\setkeys{Gin}{width=\linewidth,totalheight=\textheight,keepaspectratio}
\graphicspath{{graphics/}}

\title{STAT  245\\Homework 6}
\author{Joe Seidel}
\date{\today}


\usepackage{booktabs}
\usepackage{units}
\usepackage{fancyvrb}
\fvset{fontsize=\normalsize}
\usepackage{multicol}
\usepackage{lipsum}
\usepackage{pdfpages}
\usepackage{tikz}
\usepackage{wasysym}
\usepackage{amssymb}
\usepackage{hyperref}
\usepackage{float}
\restylefloat{table}

\hypersetup{
    colorlinks=true,
    linkcolor=blue,
    filecolor=magenta,
    urlcolor=cyan,
}


\newcommand{\doccmd}[1]{\texttt{\textbackslash#1}}% command name -- adds backslash automatically
\newcommand{\docopt}[1]{\ensuremath{\langle}\textrm{\textit{#1}}\ensuremath{\rangle}}% optional command argument
\newcommand{\docarg}[1]{\textrm{\textit{#1}}}% (required) command argument
\newenvironment{docspec}{\begin{quote}\noindent}{\end{quote}}% command specification environment
\newcommand{\docenv}[1]{\textsf{#1}}% environment name
\newcommand{\docpkg}[1]{\texttt{#1}}% package name
\newcommand{\doccls}[1]{\texttt{#1}}% document class name
\newcommand{\docclsopt}[1]{\texttt{#1}}% document class option name
\DeclareMathOperator{\proj}{proj}
\newcommand{\vct}{\mathbf}
\newcommand{\dprod}[2]{\langle \vct{#1}, \vct{#2} \rangle}
\newcommand{\pdv}[2]{\frac{\partial #1}{\partial #2}}
\DeclareMathOperator{\Var}{Var}
\newtheoremstyle{mytheoremstyle} % name
	{\topsep}		% Space above
	{\topsep}		% Space below
	{\itshape}		% Body font
	{}			% Indent amount
	{\bfseries}	% Theorem head font
	{\textnormal{:}}	% Punctuation after theorem head
	{.5em}		% Space after theorem head
	{}			%Theorem headspec
\theoremstyle{mytheoremstyle}
\newtheorem*{thm}{Thm.}

\newtheoremstyle{mylemstyle} % name
	{\topsep}		% Space above
	{\topsep}		% Space below
	{\itshape}		% Body font
	{}			% Indent amount
	{\bfseries}	% Theorem head font
	{\textnormal{:}}	% Punctuation after theorem head
	{.5em}		% Space after theorem head
	{}			%Theorem headspec
\theoremstyle{mylemstyle}
\newtheorem*{lem}{Lem.}


\newtheoremstyle{mydefstyle} % name
	{\topsep}		% Space above
	{\topsep}		% Space below
	{\normalfont}	% Body font
	{}			% Indent amount
	{\bfseries}	% Theorem head font
	{\textnormal{:}}	% Punctuation after theorem head
	{.5em}		% Space after theorem head
	{}			%Theorem headspec
\theoremstyle{mydefstyle}
\newtheorem*{mydef}{Def.}
\newtheorem*{ex}{E.g.}


\begin{document}

\maketitle
\pagenumbering{gobble}
\newpage
\pagenumbering{arabic}

\subsection{1. Sampling Distribution}
Before the midterm, we learned that for i.i.d. obserations $y_1,...,y_n \sim N(0, \sigma^2)$, $\frac{\sum_{i=1}^n(y_i-\overline{y})^2}{\sigma^2} \sim \chi_{n-1}^2$.  Review the derivation of this result in your notes.

We have $y_1 = \sigma z_i$ where $z_i \sim N(0,1)$ and i.d.d.  Therefore $\overline{y}=\sigma \overline{z}$, $(y_i - \overline{y})^2 = \sigma^2(z_i-\overline{z})^2$.

Let $A= I_n - \frac{1}{n} \mathbb{1}\mathbb{1}^T$ where $\mathbb{1} \in \mathbb{R}^{n \times 1}$.  Then

\[
\begin{pmatrix}
z_1 - \overline{z}\\
\vdots \\
z_n - \overline{z}
\end{pmatrix}
= A \cdot Z = (I - \frac{1}{n}\mathbb{1}\mathbb{1}^T)\cdot Z.
\]

By eigen decomposition,
\[ A = Q
\begin{pmatrix}
I_{n-1} & \\
& 0\\
\end{pmatrix}
Q^T.
\]

Therefore
\[ \sum_{i=1}^n(z_i - \overline{z})^2 = \|Az\|^2 = (Az)^T(Az) = z^TQ\begin{pmatrix}
I_{n-1} & \\
& 0\\
\end{pmatrix}
Q^Tz. \]

Let $\tilde{z} = Q^Tz$, then $\tilde{z} \sim N(0, Q^TIQ) \sim N(0,I)$.

Therefore
\[ \|Az\|^2 = \tilde{z}^T\begin{pmatrix}
I_{n-1} & \\
& 0\\
\end{pmatrix}\tilde{z} = \sum_{i=1}^{n-1} \tilde{z}^2 \sim \chi_{n-1}^2 \]
it follows

\[ \frac{\sum_{i=1}^n (y_i - \overline{y})^2}{\sigma^2} =\sum_{i=1}^n \frac{(y_i-\overline{y})^2}{\sigma^2} \sim \chi_{n-1}^2 \]

\subsection{2. ANOVA}
Today, we learned one-way ANOVA.  In this setting, we have independent observations $y_{ij} \sim N(\mu + \alpha_i, \sigma^2)$ for $i=1,...,n$ and $j=1,...,m$.  We also imposed the constraint $\sum_{i=1}^n\alpha_i = 0$ for the sake of identifiability.

\begin{enumerate}

\item[(a)] Under the null hypothesis that $\alpha_1 = \alpha_2= ...=\alpha_n$, together with the constraint $\sum_{i=1}^n\alpha_i = 0$, show $\alpha_i = 0$ for each $i=1,...,n$.

Since $\alpha_1=\alpha_2=...=\alpha_n$ then we have $\sum_{i=1}^n\alpha_i =n\alpha = 0$ which implies $\alpha_i=0$.

\item[(b)]  Show $\frac{\sum_{j=1}^m(y_{ij} - \overline{y}_{i.})^2}{\sigma^2} \sim \chi_{m-1}^2$ under the null.  You can use the sample distribution theorem.

Under the null $\alpha_i = 0$ for $i=1,...,n$. Therefore $y_{ij} \sim N(\mu, \sigma^2)$, i.i.d.  for $i=1,...,n$, $j=1,...,m$ and $\overline{y}_i \sim N(\mu, \frac{\sigma}{n})$.  Since $\overline{y}_i - \mu = \frac{1}{m}\sum_{j=1}^m(y_{ij}-\mu)$ then $(y_{ij} - \mu) \sim N(0, \sigma^2)$.  Applying the sampling distribution from question 1 we have

\[ \frac{\sum_{j=1}^m (y_{ij} - \overline{y}_{i.})^2}{\sigma^2} = \frac{\sum_{j=1}^m [(y_{ij} - \mu) -(\overline{y}_{i.}-\mu)]^2}{\sigma^2} \sim \chi_{m-1}^2 \]


\item Show $\frac{\sum_{i=1}^n \sum_{j=1}^m(y_{ij} - \overline{y}_{i.})^2}{\sigma^2} \sim \chi_{n(m-1)}^2$ under the null.  This is the first conclusion in the theorem you learned today.

Observe

\[ \frac{\sum_{i=1}^n \sum_{j=1}^m(y_{ij} - \overline{y}_{i.})^2}{\sigma^2} = \sum_{i=1}^n \frac{\sum_{j=1}^m(y_{ij} - \overline{y}_{i.})^2}{\sigma^2} \]

where we've found in part (b) that

\[ \frac{\sum_{j=1}^m(y_{ij} - \overline{y}_{i.})^2}{\sigma^2} \sim \chi_{m-1}^2 .\]

For each $i,j$, $y_{ij}$ are independent.  Therefore sum of $\chi_{m-1}^2$ is $\chi_{n(m-1)}^2$ e.g.

\[ \sum_{i=1}^n \sum_{j=1}^m\frac{(y_{ij} - \overline{y}_{i.})^2}{\sigma^2} \sim \chi_{n(m-1)}^2. \]

\item[(d)] Define $x_i \sqrt{m}\overline{y}_i$ and show $m\sum_{i=1}^n(\overline{y}_{i.} - \overline{y})^2 = \sum_{i=1}^n(x_i-\overline{x})^2$.

\begin{align*}
m\sum_{i=1}^n(\overline{y}_{i.} - \overline{y})^2 &= \sum_{i=1}^n(\sqrt{m}\overline{y}_{i.} - \sqrt{m}\overline{y})^2 \\
&= \sum_{i=1}^n(x_i - \frac{1}{mn}\sum_{j=1}^m\sum_{i=1}^n\sqrt{m}y_{ij})^2 \\
&= \sum_{i=1}^n(x_i - \frac{1}{n}\sum_{i=1}^n(\sqrt{m}\cdot \frac{1}{m}\sum_{j=1}^my_{ij}))^2 \\
&=\sum_{i=1}^n(x_i - \frac{1}{n}\sum_{i=1}^n\sqrt{m}\overline{y}_{i.})^2 \\
&= \sum_{i=1}^n(x_i - \frac{1}{n}\sum_{i=1}^nx_i)^2 \\
&= \sum_{i=1}^n(x_i-\overline{x})^2
\end{align*}
\end{enumerate}

\end{document}
\grid
